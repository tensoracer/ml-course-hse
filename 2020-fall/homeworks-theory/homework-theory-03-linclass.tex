\documentclass[12pt,fleqn]{article}

\usepackage{../lecture-notes/vkCourseML}

\theorembodyfont{\rmfamily}
\newtheorem{esProblem}{Задача}

\begin{document}

\title{Машинное обучение\\ФКН ВШЭ\\Теоретическое домашнее задание №3}

\date{}

\author{}

\maketitle


\begin{esProblem}
		Пусть даны выборка~$X$, состоящая из 8 объектов, и классификатор~$b(x)$, предсказывающий оценку принадлежности объекта положительному классу. Предсказания~$b(x)$ и реальные метки объектов приведены ниже:
		\begin{align*}
			&b(x_1) = 0.1, \quad  y_1 = +1,\\
			&b(x_2) = 0.8, \quad y_2 = +1,\\
			&b(x_3) = 0.2, \quad y_3 = -1,\\
			&b(x_4) = 0.25, \quad y_4 = -1,\\
			&b(x_5) = 0.9, \quad y_5 = +1,\\
			&b(x_6) = 0.3, \quad y_6 = +1,\\
			&b(x_7) = 0.6, \quad y_7 = -1,\\
			&b(x_8) = 0.95, \quad y_8 = +1.\\
		\end{align*}
    Постройте ROC-кривую и вычислите AUC-ROC для множества классификаторов~$a(x;t)$, порожденных~$b(x)$, на выборке~$X$.

\end{esProblem}

\begin{esProblem}
	Пусть дан  классификатор $b(x)$, который возвращает оценку принадлежности объекта $x$ положительному классу. Отсортируем все объекты по неубыванию ответа классификатора: $b(x_{(1)}) \le \dots \le b(x_{(\ell)})$. Обозначим истинные ответы на этих объектах через $y_{(1)}, \dots, y_{(\ell)}$.
	
	Покажите, что AUC-ROC для данной выборки будет равен вероятности того, что случайно выбранный положительный объект окажется в отсортированном списке не раньше случайно выбранного отрицательного объекта.

\end{esProblem}

\begin{esProblem}
	Пусть дана некоторая выборка $X$ и классификатор $b(x),$ возвращающий в качестве оценки принадлежности объекта  $x$ положительному классу 0 или 1 (а не некоторое вещественное число, как предполагалось на семинарах).
	\begin{enumerate}
	\item
	Постройте ROC-кривую для классификатора $b(x)$ на выборке $X$.
	\item
	Покажите, что AUC-ROC классификатора $b(x)$ на выборке $X$ может быть выражен через долю правильных ответов и полноту классификатора $a(x; t)$, получающегося при выборе некоторого порога $t \in (0; 1)$. Помимо указанных величин в формулу могут входить только величины $\ell_-, \, \ell_+, \, \ell$ (количество отрицательных, положительных и общее количество объектов в выборке  $X$ соответственно).
	\item
	Покажите, что в случае сбалансированной выборки ($\ell_- = \ell_+$) AUC-ROC классификатора $b(x)$ на выборке $X$ совпадает с долей правильных ответов классификатора при выборе некоторого порога $t \in (0;1).$
	\end{enumerate} 
\end{esProblem}

\begin{esProblem}
	В анализе данных для сравнения среднего значения некоторой величины у объектов двух выборок часто используется критерий Манна–Уитни–Уилкоксона\footnote{\href{https://en.wikipedia.org/wiki/Mann\%2dWhitney\_U\_test}{https://en.wikipedia.org/wiki/Mann–Whitney\_U\_test}}, основанный на вычислении $U$-статистики.
	\par Пусть у нас имеется выборка $X$ и классификатор $b(x)$, возвращающий оценку принадлежности объекта $x$ положительному классу. Тогда вычисление $U$-статистики для подвыборки $X$, состоящей из объектов положительного класса, производится следующим образом: объекты обеих выборок сортируются по неубыванию значения~$b(x)$, после чего каждому объекту в полученном упорядоченном ряду $x_{(1)}, \dots, x_{(\ell)}$ присваивается ранг — номер позиции $r_{(i)}$ в ряду (начиная с 1, при этом для объектов с одинаковыми значением $b(x)$ в качестве ранга присваивается среднее значение ранга для таких объектов). Тогда $U$-статистика для объектов положительного класса равна:
	$$U_+ = \sum_{\substack{i= 1 \\ y_{(i)} = +1}}^\ell r_{(i)} - \frac{\ell_+ (\ell_+ + 1)}{2}.$$
	\par Покажите, что для значения AUC-ROC классификатора $b(x)$ на выборке $X$ и $U$-статистики верно следующее соотношение:
	$$\text{AUC} = \frac{U_+}{\ell_- \ell_+}.$$
	
\end{esProblem}

\begin{esProblem}
    Позволяет ли предсказывать корректные вероятности экспоненциальная функция потерь~$L(y, z) = \exp(-yz)$?
\end{esProblem}

\begin{esProblem}
    Рассмотрим постановку оптимизационной задачи метода опорных векторов для линейно разделимой выборки:
    \begin{align*}
        \begin{cases}
            \frac{1}{2} \| w\|^2 \to \min_{w, b},\\
            y_i (\langle w, x\rangle + b) \ge 1, \quad i = \overline{1, \ell},
        \end{cases}
    \end{align*}
    а также её видоизменёный вариант для некоторого значения $t > 0$:
\begin{align*}
        \begin{cases}
            \frac{1}{2} \| w\|^2 \to \min_{w, b},\\
            y_i (\langle w, x\rangle + b) \ge t, \quad i = \overline{1, \ell}.
        \end{cases}
    \end{align*}
     Покажите, что разделяющие гиперплоскости, получающиеся в результате решения каждой из этих задач, совпадают.
        
\end{esProblem}

%\begin{esProblem}
%    Пусть мы решили двойственную задачу SVM и получили
%    оптимальные значения~$(\lambda_1, \dots, \lambda_{\ell})$,
%    где~$\lambda_5 = C/3$, $\lambda_2 = 0$.
%    Выразите оптимальное значение порога $b$ для~прямой задачи через
%    найденное решение~$(\lambda_1, \dots, \lambda_{\ell})$
%    двойственной задачи.
%\end{esProblem}

\begin{esProblem}
    Вычислите градиент $\frac{\partial}{\partial w}L(x, y; w)$ логистической функции потерь для случая линейного классификатора
    $$L(x, y; w) = \log (1 + \exp(-y \, \langle w, x\rangle))$$
    и упростите итоговое выражение таким образом, чтобы в нём участвовала сигмоидная функция 
    $$\sigma(z) = \frac{1}{1 + \exp(-z)}.$$
    При решении данной задачи вам может понадобиться следующий факт (убедитесь, что он действительно выполняется):
    $$\sigma'(z) = \sigma(z) (1- \sigma(z)).$$ 
\end{esProblem}

\begin{esProblem}
    Ответьте на следующие вопросы:
    \begin{enumerate}
    \item Почему в общем случае распределение $p(y|x)$ для некоторого объекта $x \in \mathbb{X}$ отличается от вырожденного ($p(y|x) \in \{0,1\}$)?
    \item Почему логистическая регрессия позволяет предсказывать корректные вероятности принадлежности объекта классам?
    \item Рассмотрим оптимизационную задачу из варианта SVM для линейно разделимых выборок. Всегда ли в обучающей выборке существует объект $x_i$, для которого выполнено $y_i (\langle w, x_i \rangle + b) = 1$? Почему?
    \item С какой целью в постановке оптимизационной задачи SVM для линейно неразделимых выборок вводятся переменные $\xi_i, \, i = \overline{1, \ell}?$
    \end{enumerate}
\end{esProblem}

\end{document}

