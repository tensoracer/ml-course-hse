\documentclass[12pt,fleqn]{article}
\usepackage{../lecture-notes/vkCourseML}
\usepackage{gensymb}
\hypersetup{unicode=true}
%\usepackage[a4paper]{geometry}
\usepackage[hyphenbreaks]{breakurl}

\interfootnotelinepenalty=10000

\begin{document}
\title{Машинное обучение, ФКН ВШЭ\\Семинар №20}
\author{}
\date{}
\maketitle

\section{Обучение метрик}

В методе~$k$ ближайших соседей не так много параметров~--- число соседей, функция расстояния,
ядро и его ширина.
Можно выбирать метрику из числа известных~--- например, из евклидовой, манхэттенской и косинусной.
Эти метрики фиксированы и никак не могут быть подстроены под особенности данных.
Кажется, что обучение метрики под выборку могло бы увеличить число степеней свободы у метрических методов
и позволить добиваться более высокого качества.
Например, масштаб признаков может существенно влиять на их важность при вычислении расстояний.

Рассмотрим простой пример.
Допустим, решается задача определения пола человека по двум признакам:
росту~(в сантиметрах, принимает значения примерно от $150$ до $200$)
и уровню экспрессии гена SRY~(безразмерная величина от нуля до единицы;
у мужчин ближе к единице, у женщин ближе к нулю).
Обучающая выборка состоит из двух объектов:
$x_1 = (180, 0.2)$, девочка и $x_2 = (173, 0.9)$, мальчик.
Требуется классифицировать новый объект $u = (178, 0.85)$.
Воспользуемся классификатором одного ближайшего соседа.
Евклидовы расстояния от $u$ до объектов обучения равны~$\rho(u, x_1) \approx 2.1$ и $\rho(u, x_2) \approx 5$.
Мы признаем новый объект девочкой, хотя это не так~--- высокий уровень экспрессии гена SRY
позволяет с уверенностью сказать, что это мальчик.
Из-за сильных различий в масштабе признаков уровень экспрессии практически не учитывается
при классификации, что совершенно неправильно.

Удобнее всего обучать метрику через линейные преобразования признаков:
\[
    \rho(x, z)
    =
    \| Ax - Az \|^2
    =
    (x - z)^T A^T A (x - z),
\]
где~$A \in \RR^{n \times d}$~--- матрица, которую можно подбирать.
По сути, обучение линейного преобразования равносильно настройке параметра~$\Sigma$
в метрике Махалонобиса:
\[
    \rho(x, z)
    =
    (x - z)^T \Sigma^{-1} (x - z),
\]
если положить~$A = \Sigma^{-1/2}$.
Нелинейные методы часто сводятся к обучению линейных
в новом признаковом пространстве~(т.е.~$\rho(x, z) = \|A \phi(x) - A \phi(z)\|^2$)
либо путём ядрового перехода в линейном методе~\cite{kulis12survey}.

Мы разберём два подхода к обучению расстояния Махалонобиса.

\subsection{Neighbourhood Components Analysis}

Метод NCA~\cite{goldberger05nca} выбирает метрику так, чтобы для каждого объекта
ближайшими оказывались объекты его же класса.
Рассмотрим объект~$x_i$ и рассмотрим следующий эксперимент:
мы выбираем из оставшейся выборки случайный объект~$x_j$ и относим~$x_i$ к классу~$y_j$.
Зададим вероятности через расстояния между объектами:
\[
    p_{ij}
    =
    \begin{cases}
        \frac{
            \exp(-\|Ax_i - Ax_j\|^2)
        }{
            \sum_{k \neq i} \exp(-\|Ax_i - Ax_k\|^2)
        },
        &i \neq j\\
        0, &i = j
    \end{cases}
\]
Можно вычислить вероятность того, что объект~$x_i$ будет отнесён к правильному классу.
Если обозначить через~$C_i = \{j \cond y_i = y_j\}$ множество индексов объектов того же класса,
то данная вероятность равна
\[
    p_i
    =
    \sum_{j \in C_i}
    p_{ij}.
\]
Будем максимизировать матожидание количества верно классифицированных объектов:
\[
    Q(A)
    =
    \sum_{i = 1}^{\ell} p_i
    \to
    \max_{A}
\]
Этот функционал можно продифференцировать по~$A$:
\[
    \frac{\partial Q}{\partial A}
    =
    2A
    \sum_{i} \left(
        p_i
        \sum_{k}
            p_{ik} (x_i - x_k) (x_i - x_k)^T
        -
        \sum_{j \in C_i}
            p_{ij} (x_i - x_j) (x_i - x_j)^T
    \right).
\]
Далее матрицу~$A$ можно обучать любым градиентным методом.

Отметим, что метод NCA можно использовать и для ускорения поиска ближайших соседей.
Если взять матрицу~$A \in \RR^{n \times d}$ с небольшой первой размерностью~$n$,
то она будет переводить объекты в компактные представления,
евклидова метрика на которых позволяет хорошо отделять классы друг от друга.

\subsection{Large margin nearest neighbor}

Метод LMNN~\cite{weinberger06lmnn} пытается обучить метрику так,
чтобы~$k$ ближайших соседей каждого объекта относились к нужному классу,
а объекты из других классов отделялись с большим отступом.
Попытаемся ввести соответствующий функционал.

Определим для каждого объекта~$x_i$ набор из~$k$ целевых соседей~---
объектов, расстояние до которых должно оказаться минимальным.
В простейшем варианте это могут быть ближайшие~$k$ объектов
из этого же класса, но можно выбирать их и иначе.
Введём индикатор~$\eta_{ij} \in \{0, 1\}$,
который равен единице, если объект~$x_j$ является целевым соседом для~$x_i$.

Выше мы поставили перед собой две цели: минимизировать расстояние до целевых соседей
и максимизировать расстояние до объектов других классов.
Суммарное расстояние до целевых соседей можно вычислить как
\[
    \sum_{i \neq j}
        \eta_{ij}
        \| A x_i - A x_j \|^2.
\]
Для объектов других классов будем требовать, чтобы расстояние до них хотя бы на единицу
превосходило расстояния до целевых соседей:
\[
    \sum_{i = 1}^{\ell}
    \sum_{j \neq i}
    \sum_{\substack{m \neq i \\ m \neq j}}
        \eta_{ij}
        [y_m \neq y_i]
        \max(0, 1 + \|Ax_i - Ax_j\|^2 - \|Ax_i - Ax_m\|^2).
\]
Суммируя эти два выражения, получим итоговый функционал:
\begin{align*}
    \sum_{i \neq j}
        &\eta_{ij}
        \| A x_i - A x_j \|^2+\\
    &+
    C
    \sum_{i = 1}^{\ell}
    \sum_{j \neq i}
    \sum_{\substack{m \neq i \\ m \neq j}}
        \eta_{ij}
        [y_m \neq y_i]
        \max(0, 1 + \|Ax_i - Ax_j\|^2 - \|Ax_i - Ax_m\|^2)
    \to
    \min_{A}
\end{align*}
Данную задачу можно свести к стандартной задаче с линейным функционалом
и ограничениями на неотрицательную определённость матрицы
и решена стандартными солверами.

\begin{thebibliography}{1}
\bibitem{kulis12survey}
    \emph{Kulis, B.} (2012).
    Metric Learning: A Survey.~//
    Foundations and Trends in Machine Learning.

\bibitem{goldberger05nca}
    \emph{Goldberger J., Hinton G., Roweis S., Salakhutdinov R.} (2005).
    Neighbourhood Components Analysis.~//
    Advances in Neural Information Processing Systems.

\bibitem{weinberger06lmnn}
    \emph{Weinberger, K. Q.; Blitzer J. C.; Saul L. K.} (2006).
    Distance Metric Learning for Large Margin Nearest Neighbor Classification.~//
    Advances in Neural Information Processing Systems.
\end{thebibliography}

\end{document}
