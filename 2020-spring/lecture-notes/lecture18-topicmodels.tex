\documentclass[12pt,fleqn]{article}
\usepackage{vkCourseML}
\usepackage{gensymb}
\hypersetup{unicode=true}
%\usepackage[a4paper]{geometry}
\usepackage[hyphenbreaks]{breakurl}

\interfootnotelinepenalty=10000

\begin{document}
\title{Лекция 18\\Тематическое моделирование}
\author{Е.\,А.\,Соколов\\ФКН ВШЭ}
\maketitle

Мы уже не раз сталкивались с идеей поиска~\emph{представлений} для объектов~---
например, слов~(word2vec) или изображений~(последние слои свёрточных сетей).
В этой лекции мы обсудим задачу поиска представлений для текстов.

Текст может кодироваться с помощью вектора, каждый элемент которого соответствует одному слову из словаря,
а его значение вычисляется как число вхождений этого слова в текст или, например, как TF-IDF.
Впрочем, такой подход никак не учитывает наличие синонимов~(было бы полезно объединять слова по смыслу),
многозначных слов~(которые, наоборот, не следует объединять в одну координату)
и других смысловых особенностей.
Чтобы работать на уровне смыслов, предположим, что существует~$T$ \emph{тем}~(topics),
и каждый документ~$x_d$ характеризуется вектором~$\theta_d \in \RR^T$.
Элементы этого вектора должны описывать наличие той или иной темы в данном документе.
Далее можно рассматривать эти векторы как новые признаковые описания документов,
характеризующие его уже на уровне смыслов, а не слов.
Темой называется вектор~$\phi_t \in \RR^W$, где~$W$~--- размер словаря;
этот вектор должен характеризовать принадлежность каждого слова к данной теме.
Ниже мы рассмотрим несколько подходов к построению таких~\emph{тематических моделей}

\section{Latent semantic analysis (LSA)}

Построим матрицу~$X \in \RR^{D \times W}$, где~$D$~--- число документов,~$W$~--- размер словаря.
С помощью сингулярного разложения найдём лучшую аппроксимацию~(в смысле квадратичного отклонения)
ранга~$T$:
\[
    X
    =
    \Theta \Phi,
    \quad
    \Theta \in \RR^{D \times T},\ 
    \Phi \in \RR^{T \times W}.
\]
Строки матрицы~$\Theta$ можно интерпретировать как распределения тем в документах,
столбцы матрицы~$\Phi$~--- как распределения слов в темах.
Заметим, что эти векторы не являются распределениями в прямом смысле, поскольку
их элементы могут быть отрицательными.
Такие представления могут быть полезны для понижения размерности или для учёта
смысловых близостей слов, но не поддаются интерпретации.

\section{Probabilistic latent semantic analysis (PLSA)}

Будем считать, что каждый документ~$x_d$ описывается распределением~$p(t \cond d) = \theta_{td}$,
а каждая тема~--- распределением~$p(w \cond t) = \phi_{wt}$.
Тогда совместное распределение на словах и документах можно записать как
\[
    p(w, d)
    =
    p(d) p(w \cond d)
    =
    p(d)
    \sum_{t = 1}^{T}
        p(w \cond t) p(t \cond d).
\]
Здесь мы, по сути, ввели скрытую переменную~$t$, которая показывает,
из какой темы было сгенерировано слово~$w$ документа~$x_d$.
Согласно данной модели, документ~$x_d$ генерируется по следующей схеме:
\begin{enumerate}
    \item Выбираем тему~$t \sim p(t \cond d)$;
    \item Выбираем слово из данной темы~$w \sim p(w \cond t)$;
    \item Повторяем шаги 1 и 2, если текст не достиг требуемой длины.
\end{enumerate}

Чтобы записать правдоподобие, следует смотреть на набор документов
как на выборку пар~<<документ-слово>>.
Неполное правдоподобие данной модели имеет вид
\[
    \sum_{d = 1}^{D}
    \sum_{j = 1}^{|x_d|}
        \log \sum_{t = 1}^{T} \phi_{w_{dj}t} \theta_{td},
\]
где~$w_{dj}$~--- $j$-е по порядку слово из документа~$x_d$.
Если для каждой пары~<<документ-слово>>~$(d, w_{dj})$ известно,
из какой темы~$t_{dj}$ оно сгенерировано, то можно записать полное правдоподобие:
\[
    \sum_{d = 1}^{D}
    \sum_{j = 1}^{|x_d|}
    \sum_{t = 1}^{T}
        [t_{dj} = t]
        \log \phi_{w_{dj}t} \theta_{td}.
\]

Мы уже знаем, что для обучения таких моделей можно воспользоваться EM-алгоритмом.
На E-шаге оценим апостериорные распределения на скрытых переменных по формуле Байеса:
\[
    p(t_{dj} \cond d, w_{dj})
    =
    \frac{
        p(w_{dj} \cond t_{dj}) p(t_{dj} \cond d)
    }{
        p(w_{dj} \cond d)
    }
    =
    \frac{
        \phi_{w_{dj}t_{dj}} \theta_{t_{dj} d}
    }{
        p(w_{dj} \cond d)
    }
\]

На M-шаге найдём максимум матожидания полного правдоподобия по скрытым переменным:
\begin{align*}
    &\phi_{wt}
    =
    \frac{
        \sum_{d = 1}^{D}
            n_{dw} p(t \cond d, w)
    }{
        \sum_{w = 1}^{W}
        \sum_{d = 1}^{D}
            n_{dw} p(t \cond d, w)
    };\\
    &\theta_{td}
    =
    \frac{
        \sum_{w = 1}^{W}
            n_{dw} p(t \cond d, w)
    }{
        \sum_{t = 1}^{T}
        \sum_{w = 1}^{W}
            n_{dw} p(t \cond d, w)
    }.
\end{align*}
Здесь~$n_{dw}$~--- число вхождений слова~$w$ в документ~$x_d$.

Полученная в итоге работы~EM-алгоритма модель будет интепретируемой~---
можно изучать, насколько сильно та или иная тема представлена в документе,
или насколько то или иное слово характерно для темы.

\section{Latent Dirichlet Allocation (LDA)}

Модель PLSA не является полной~--- распределения~$\phi_t$ и~$\theta_d$
считаются параметрами, не предполагается, что они тоже генерируются из некоторых распределений.
Значит, с помощью данной модели не получится описать процесс порождения
набора документов~<<с нуля>>.
Более того, в PLSA отсутствует регуляризация, из-за чего модель
может слишком сильно подогнаться под данные на небольших выборках.

Чтобы устранить два указанных недостатка, введём априорные распределения на
векторах~$\phi_t$ и~$\theta_d$.
Для этого хорошо подходит симметричное распределение Дирихле,
которое задано на множестве всех дискретных распределений с фиксированным числом исходов:
\begin{align*}
    &\phi_t \sim \text{Dir}(\alpha);\\
    &\theta_d \sim \text{Dir}(\beta);\\
    &\text{Dir}(x_1, \dots, x_n; \alpha)
    =
    \frac{\Gamma(\alpha n)}{\Gamma(\alpha)^n}
    \prod_{i = 1}^{n}
        x_i^{\alpha - 1}.
\end{align*}

После введения априорного распределения модель становится достаточно сложной для вывода.
Как правило, для её обучения применяют техники вариационного вывода или семплирование Гиббса.

\end{document}
